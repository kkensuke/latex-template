\chapter{Introduction}
\lipsum[1-6]\cite{aharonov1998quantum}


\section{Your boxes}
An example theorem is shown in theorem~\ref{th:pnt}, and there is
example~\ref{ex:bertrand}.
\begin{Theorem}{Prime Number Theorem (PNT)}{pnt}
  \begin{equation*}
    \pi(x)\sim\frac{x}{\log x}
  \end{equation*}
\end{Theorem}

\begin{Example}{Generalisation of Bertrand's Postulate}{bertrand}
  Let $\varepsilon>0$. Prove that there exist a prime between $n$ and
  $(1+\varepsilon)n$ for all large $n$, in particular there always exist a
  prime between $n$ and $2n$ for $n>1$.
\end{Example}