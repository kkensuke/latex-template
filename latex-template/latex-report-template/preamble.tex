% set margins
\usepackage[top=15truemm,bottom=15truemm,left=15truemm,right=15truemm]{geometry}
\usepackage[dvipdfmx]{graphicx,hyperref,xcolor}
% math
\usepackage{amsmath,amsthm,amssymb,mathtools,mathrsfs}
% physics
\usepackage{physics}
% itembox
\usepackage{ascmac}
% algorithm
\usepackage{algorithm,algorithmic}
% vector graphics
\usepackage{tikz}
% author and affiliation
\usepackage{authblk}
% comment
\usepackage{comment}
% image position
\usepackage{here}
% Align table columns on decimal point
\usepackage{dcolumn}

% style setting
% ---------------------------------------------------------------------------- %
\allowdisplaybreaks[1]
\renewcommand\Authfont{\fontsize{14}{14.4}\selectfont}
\renewcommand\Affilfont{\fontsize{10}{10.8}\itshape}
\renewcommand{\baselinestretch}{1.25}
% ---------------------------------------------------------------------------- %


% number figures, tables and equations within the sections
% ---------------------------------------------------------------------------- %
\numberwithin{equation}{section}
\numberwithin{figure}{section}
\numberwithin{table}{section}
% ---------------------------------------------------------------------------- %


% Logic and proofs
% ---------------------------------------------------------------------------- %
\newtheorem{theorem}{Theorem}[section]
\newtheorem{corollary}{Corollary}[theorem]
\newtheorem{lemma}[theorem]{Lemma}
\newtheorem{proposition}[theorem]{Proposition}

\theoremstyle{definition}
\newtheorem{definition}{Definition}[section]
\newtheorem{example}{Example}[section]
\newtheorem{exercise}{Exercise}[section]

\theoremstyle{remark}
\newtheorem{remark}{Remark}[section]
% ---------------------------------------------------------------------------- %


% Latin abbreviations
% ---------------------------------------------------------------------------- %
\newcommand{\etal}{\textit{et al. }}
\newcommand{\eg}{\textit{e.g. }}
\newcommand{\cf}{\textit{c.f. }}
\newcommand{\ie}{\textit{i.e. }}
\newcommand{\etc}{\textit{etc. }}
% ---------------------------------------------------------------------------- %


% math
% ---------------------------------------------------------------------------- %
% limit type
\DeclareMathOperator*{\argmin}{arg~min}
\DeclareMathOperator*{\argmax}{arg~max}

% \boldsymbol
\newcommand{\bx}{\vb*{x}}
\newcommand{\bth}{\vb*{\theta}}

% mathbb
\newcommand{\bbA}{\mathbb{A}}
\newcommand{\bbB}{\mathbb{B}}
\newcommand{\bbC}{\mathbb{C}}
\newcommand{\bbD}{\mathbb{D}}
\newcommand{\bbE}{\mathbb{E}}
\newcommand{\bbF}{\mathbb{F}}
\newcommand{\bbG}{\mathbb{G}}
\newcommand{\bbH}{\mathbb{H}}
\newcommand{\bbI}{\mathbb{I}}
\newcommand{\bbJ}{\mathbb{J}}
\newcommand{\bbK}{\mathbb{K}}
\newcommand{\bbL}{\mathbb{L}}
\newcommand{\bbM}{\mathbb{M}}
\newcommand{\bbN}{\mathbb{N}}
\newcommand{\bbO}{\mathbb{O}}
\newcommand{\bbP}{\mathbb{P}}
\newcommand{\bbQ}{\mathbb{Q}}
\newcommand{\bbR}{\mathbb{R}}
\newcommand{\bbS}{\mathbb{S}}
\newcommand{\bbT}{\mathbb{T}}
\newcommand{\bbU}{\mathbb{U}}
\newcommand{\bbV}{\mathbb{V}}
\newcommand{\bbW}{\mathbb{W}}
\newcommand{\bbX}{\mathbb{X}}
\newcommand{\bbY}{\mathbb{Y}}
\newcommand{\bbZ}{\mathbb{Z}}

% \mathcal
\newcommand{\calA}{\mathcal{A}}
\newcommand{\calB}{\mathcal{B}}
\newcommand{\calC}{\mathcal{C}}
\newcommand{\calD}{\mathcal{D}}
\newcommand{\calE}{\mathcal{E}}
\newcommand{\calF}{\mathcal{F}}
\newcommand{\calG}{\mathcal{G}}
\newcommand{\calH}{\mathcal{H}}
\newcommand{\calI}{\mathcal{I}}
\newcommand{\calJ}{\mathcal{J}}
\newcommand{\calK}{\mathcal{K}}
\newcommand{\calL}{\mathcal{L}}
\newcommand{\calM}{\mathcal{M}}
\newcommand{\calN}{\mathcal{N}}
\newcommand{\calO}{\mathcal{O}}
\newcommand{\calP}{\mathcal{P}}
\newcommand{\calQ}{\mathcal{Q}}
\newcommand{\calR}{\mathcal{R}}
\newcommand{\calS}{\mathcal{S}}
\newcommand{\calT}{\mathcal{T}}
\newcommand{\calU}{\mathcal{U}}
\newcommand{\calV}{\mathcal{V}}
\newcommand{\calW}{\mathcal{W}}
\newcommand{\calX}{\mathcal{X}}
\newcommand{\calY}{\mathcal{Y}}
\newcommand{\calZ}{\mathcal{Z}}

% multiplicative group
\newcommand{\Zx}{\Z^\times}
\newcommand{\Qx}{\Q^\times}
\newcommand{\Rx}{\R^\times}
\newcommand{\Cx}{\C^\times}

% non-negative
\newcommand{\Znn}{\Z_{\ge0}}
\newcommand{\Qnn}{\Q_{\ge0}}
\newcommand{\Rnn}{\R_{\ge0}}

% positive
\newcommand{\Zpo}{\Z_{>0}}
\newcommand{\Qpo}{\Q_{>0}}
\newcommand{\Rpo}{\R_{>0}}

% mathrm
\newcommand{\haar}{\mathrm{Haar}}

% MathOperator
\DeclareMathOperator{\sgn}{sgn}
\DeclareMathOperator{\sign}{sign}
\DeclareMathOperator{\Supp}{Supp}
\DeclareMathOperator{\diag}{diag}
\DeclareMathOperator{\E}{E}
\DeclareMathOperator{\Var}{Var}
\DeclareMathOperator{\Cov}{Cov}
\DeclareMathOperator{\poly}{poly}
\DeclareMathOperator{\SWAP}{SWAP}
\DeclareMathOperator{\Hom}{Hom}
\DeclareMathOperator{\Aut}{Aut}
\DeclareMathOperator{\End}{End}

% others
\renewcommand{\bar}[1]{\overline{#1}}
\newcommand{\combi}[2]{{}_{#1}\text{C}_{#2}}
\newcommand{\dg}{^\dagger}
\newcommand{\fa}{{}^\forall}
\newcommand{\ex}{{}^\exists}
\newcommand{\pd}{\partial}
\newcommand{\then}{\;\Longrightarrow\;}
% ---------------------------------------------------------------------------- %


% physics
% ---------------------------------------------------------------------------- %
% matrix
\newcommand{\paulix}{ \mqty(\pmat{1}) }
\newcommand{\pauliy}{ \mqty(\pmat{2}) }
\newcommand{\pauliz}{ \mqty(\pmat{3}) }

\newcommand{\rx}[1]{ \mqty( \cos \frac{#1}{2} & -\sin\frac{#1}{2}  \\ \sin \frac{#1}{2}  & \cos\frac{#1}{2} ) }
\newcommand{\ry}[1]{ \mqty( \cos \frac{#1}{2} & -i\sin\frac{#1}{2} \\ i\sin \frac{#1}{2} & \cos\frac{#1}{2} ) }
\newcommand{\rz}[1]{ \mqty( \exp -i{#1}/2      & 0                  \\ 0                 & \exp i{#1}/2     ) }
\newcommand{\rot}[1]{ \mqty( \cos {#1} & -\sin {#1} \\ \sin {#1} & \cos {#1} ) }

% sin cos
\newcommand{\sif}[2]{\sin\qty(\frac{#1}{#2})}
\newcommand{\cof}[2]{\cos\qty(\frac{#1}{#2})}

% others
\newcommand{\up}{\uparrow} %spin up
\newcommand{\down}{\downarrow} %spin down
\newcommand{\szero}{\mqty( 1 \\ 0 )} %spin zero
\newcommand{\sone}{\mqty( 0 \\ 1 )} %spin one
\newcommand{\otn}[2]{^{\otimes{#1}}}
\newcommand{\kten}[2]{\ket{#1}\otimes\ket{#2}}
\newcommand{\bten}[2]{\bra{#1}\otimes\bra{#2}}
% ---------------------------------------------------------------------------- %

% ref; https://github.com/SSoelvsten/latex-preamble-and-examples/tree/main/documents