% set margins
\usepackage[top=25truemm,bottom=20truemm,left=20truemm,right=20truemm]{geometry}
\usepackage[dvipdfmx]{graphicx,hyperref,color}
% math
\usepackage{amsmath,amsthm,amssymb,mathtools,mathrsfs}
% physics
\usepackage{physics}
% itembox
\usepackage{ascmac}
% algorithm
\usepackage{algorithm,algorithmic}
% vector graphics
\usepackage{tikz}
% comment
\usepackage{comment}
% dummy text
\usepackage{lipsum}
% image position
\usepackage{here}
% Align table columns on decimal point
\usepackage{dcolumn}
% author
\usepackage{authblk}
% url
\usepackage{url}
% color
\usepackage{xcolor}
% color links
\hypersetup{
    colorlinks=true,
    linkcolor=blue,
    %urlcolor=cyan,
}
% show line number
\usepackage{lineno}
\linenumbers % comment out to hide line number
% make header
\usepackage{fancyhdr}
\pagestyle{fancy}
\lhead{\leftmark}
\rhead{}
\cfoot{\thepage} % remove page number at the bottom
\renewcommand{\chaptermark}[1]{\markboth{Chpater \thechapter.\ #1}{}}
% \renewcommand{\chaptermark}[1]{\markboth{第\ \normalfont\thechapter\ 章~#1}{}} % leftmark for japanese
% \renewcommand{\sectionmark}[1]{\markright{\thesection #1}{}} % rightmark


% style setting
% ---------------------------------------------------------------------------- %
\allowdisplaybreaks[1] % allow page breaks in equations. takes an integer from 0-4. 4 is the most permissive.
\renewcommand\Authfont{\fontsize{14}{14.4}\selectfont}
\renewcommand\Affilfont{\fontsize{10}{10.8}\itshape}
\renewcommand{\baselinestretch}{1.25}
% ---------------------------------------------------------------------------- %


% number figures, tables and equations within the sections
% ---------------------------------------------------------------------------- %
\numberwithin{equation}{section}
\numberwithin{figure}{section}
\numberwithin{table}{section}
% ---------------------------------------------------------------------------- %


% Logic and proofs
% ---------------------------------------------------------------------------- %
\newtheorem{theorem}{Theorem}[section]
\newtheorem{corollary}{Corollary}[theorem]
\newtheorem{lemma}[theorem]{Lemma}
\newtheorem{proposition}[theorem]{Proposition}

\theoremstyle{definition}
\newtheorem{definition}{Definition}[section]
\newtheorem{example}{Example}[section]
\newtheorem{exercise}{Exercise}[section]

\theoremstyle{remark}
\newtheorem{remark}{Remark}[section]
% ---------------------------------------------------------------------------- %


% Latin abbreviations
% ---------------------------------------------------------------------------- %
\newcommand{\etal}{\text{et al.}}
\newcommand{\eg}{\text{e.g. }}
\newcommand{\cf}{\text{c.f. }}
\newcommand{\ie}{\text{i.e. }}
\newcommand{\etc}{\text{etc.}}
% ---------------------------------------------------------------------------- %


% math
% ---------------------------------------------------------------------------- %
% mathbb capital letters
\newcommand{\bbA}{\mathbb{A}}
\newcommand{\bbB}{\mathbb{B}}
\newcommand{\bbC}{\mathbb{C}}
\newcommand{\bbD}{\mathbb{D}}
\newcommand{\bbE}{\mathbb{E}}
\newcommand{\bbF}{\mathbb{F}}
\newcommand{\bbG}{\mathbb{G}}
\newcommand{\bbH}{\mathbb{H}}
\newcommand{\bbI}{\mathbb{I}}
\newcommand{\bbJ}{\mathbb{J}}
\newcommand{\bbK}{\mathbb{K}}
\newcommand{\bbL}{\mathbb{L}}
\newcommand{\bbM}{\mathbb{M}}
\newcommand{\bbN}{\mathbb{N}}
\newcommand{\bbO}{\mathbb{O}}
\newcommand{\bbP}{\mathbb{P}}
\newcommand{\bbQ}{\mathbb{Q}}
\newcommand{\bbR}{\mathbb{R}}
\newcommand{\bbS}{\mathbb{S}}
\newcommand{\bbT}{\mathbb{T}}
\newcommand{\bbU}{\mathbb{U}}
\newcommand{\bbV}{\mathbb{V}}
\newcommand{\bbW}{\mathbb{W}}
\newcommand{\bbX}{\mathbb{X}}
\newcommand{\bbY}{\mathbb{Y}}
\newcommand{\bbZ}{\mathbb{Z}}

% \mathrm capital letters
\newcommand{\rmA}{\mathrm{A}}
\newcommand{\rmB}{\mathrm{B}}
\newcommand{\rmC}{\mathrm{C}}
\newcommand{\rmD}{\mathrm{D}}
\newcommand{\rmE}{\mathrm{E}}
\newcommand{\rmF}{\mathrm{F}}
\newcommand{\rmG}{\mathrm{G}}
\newcommand{\rmH}{\mathrm{H}}
\newcommand{\rmI}{\mathrm{I}}
\newcommand{\rmJ}{\mathrm{J}}
\newcommand{\rmK}{\mathrm{K}}
\newcommand{\rmL}{\mathrm{L}}
\newcommand{\rmM}{\mathrm{M}}
\newcommand{\rmN}{\mathrm{N}}
\newcommand{\rmO}{\mathrm{O}}
\newcommand{\rmP}{\mathrm{P}}
\newcommand{\rmQ}{\mathrm{Q}}
\newcommand{\rmR}{\mathrm{R}}
\newcommand{\rmS}{\mathrm{S}}
\newcommand{\rmT}{\mathrm{T}}
\newcommand{\rmU}{\mathrm{U}}
\newcommand{\rmV}{\mathrm{V}}
\newcommand{\rmW}{\mathrm{W}}
\newcommand{\rmX}{\mathrm{X}}
\newcommand{\rmY}{\mathrm{Y}}
\newcommand{\rmZ}{\mathrm{Z}}

% \mathcal capital letters
\newcommand{\calA}{\mathcal{A}}
\newcommand{\calB}{\mathcal{B}}
\newcommand{\calC}{\mathcal{C}}
\newcommand{\calD}{\mathcal{D}}
\newcommand{\calE}{\mathcal{E}}
\newcommand{\calF}{\mathcal{F}}
\newcommand{\calG}{\mathcal{G}}
\newcommand{\calH}{\mathcal{H}}
\newcommand{\calI}{\mathcal{I}}
\newcommand{\calJ}{\mathcal{J}}
\newcommand{\calK}{\mathcal{K}}
\newcommand{\calL}{\mathcal{L}}
\newcommand{\calM}{\mathcal{M}}
\newcommand{\calN}{\mathcal{N}}
\newcommand{\calO}{\mathcal{O}}
\newcommand{\calP}{\mathcal{P}}
\newcommand{\calQ}{\mathcal{Q}}
\newcommand{\calR}{\mathcal{R}}
\newcommand{\calS}{\mathcal{S}}
\newcommand{\calT}{\mathcal{T}}
\newcommand{\calU}{\mathcal{U}}
\newcommand{\calV}{\mathcal{V}}
\newcommand{\calW}{\mathcal{W}}
\newcommand{\calX}{\mathcal{X}}
\newcommand{\calY}{\mathcal{Y}}
\newcommand{\calZ}{\mathcal{Z}}

% mathrm
\newcommand{\const}{\mathrm{const}}
\newcommand{\hc}{\mathrm{h.c.}}
\newcommand{\lhs}{\mathrm{(LHS)}}
\newcommand{\rhs}{\mathrm{(RHS)}}
\newcommand{\Haar}{\mathrm{Haar}}
\newcommand{\poly}{\mathrm{poly}}
\newcommand{\SWAP}{\mathrm{SWAP}}
\newcommand{\CNOT}{\mathrm{CNOT}}

% MathOperator
\DeclareMathOperator*{\argmin}{arg~min}
\DeclareMathOperator*{\argmax}{arg~max}
\DeclareMathOperator{\sgn}{sgn}
\DeclareMathOperator{\sign}{sign}
\DeclareMathOperator{\Supp}{Supp}
\DeclareMathOperator{\diag}{diag}
\DeclareMathOperator{\E}{E}
\DeclareMathOperator{\Var}{Var}
\DeclareMathOperator{\Cov}{Cov}
\DeclareMathOperator{\Hom}{Hom}
\DeclareMathOperator{\Aut}{Aut}
\DeclareMathOperator{\End}{End}

% others
\renewcommand{\bar}[1]{\overline{#1}}
\newcommand{\combi}[2]{{}_{#1}\text{C}_{#2}}
\newcommand{\dg}{^\dagger}
\newcommand{\fa}{{}^\forall}
\newcommand{\ex}{{}^\exists}
\newcommand{\pd}{\partial}
\newcommand{\vt}{\,\|\,}
\newcommand{\T}{^\mathsf{T}}
\newcommand{\bs}{\boldsymbol}
\newcommand{\ot}{\otimes}
\newcommand{\otn}[1]{^{\otimes {#1}}}
\newcommand{\kten}[2]{\ket{#1}\otimes\ket{#2}}
\newcommand{\bten}[2]{\bra{#1}\otimes\bra{#2}}
\newcommand{\memo}[1]{\textcolor{red}{(#1)}}
\renewcommand{\th}{\theta}
% ---------------------------------------------------------------------------- %


% physics
% ---------------------------------------------------------------------------- %
% matrix
\newcommand{\paulii}{\mqty[\pmat{0}]}
\newcommand{\paulix}{\mqty[\pmat{1}]}
\newcommand{\pauliy}{\mqty[\pmat{2}]}
\newcommand{\pauliz}{\mqty[\pmat{3}]}

\newcommand{\rx}[1]{ \mqty[\cos \frac{#1}{2} & -\sin\frac{#1}{2}  \\ \sin \frac{#1}{2}  & \cos\frac{#1}{2} ]}
\newcommand{\ry}[1]{ \mqty[\cos \frac{#1}{2} & -i\sin\frac{#1}{2} \\ i\sin \frac{#1}{2} & \cos\frac{#1}{2} ]}
\newcommand{\rz}[1]{ \mqty[\exp -i{#1}/2     & 0                  \\ 0                  & \exp i{#1}/2     ]}
\newcommand{\rot}[1]{\mqty[\cos {#1}         & -\sin {#1}         \\ \sin {#1}          & \cos {#1} ]}

% sin cos
\newcommand{\sif}[2]{\sin\qty(\frac{#1}{#2})}
\newcommand{\cof}[2]{\cos\qty(\frac{#1}{#2})}

% spin
\newcommand{\up}{\uparrow} %spin up
\newcommand{\down}{\downarrow} %spin down
\newcommand{\szero}{\mqty[1 \\ 0]} %spin zero
\newcommand{\sone}{\mqty[0 \\ 1]} %spin one
% ---------------------------------------------------------------------------- %

% ref: https://github.com/SSoelvsten/latex-preamble-and-examples/tree/main/documents